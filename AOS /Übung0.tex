\section{Übung 0 - AOS}


\subsection{Aufgabe 1 a;}
\begin{itemize}
    \item Flugzeug: 
    \begin{itemize}
        \item Fluggefühl
        \item kundenbezugene Wünsche an Getränken, Essen, etc.
    \end{itemize}
    \item Smartphone:
    \begin{itemize}
        \item Benutzerfreundlichkeit
        \item Design
        \item Preis
    \end{itemize}
    \item Betriebssystem:
    \begin{itemize}
        \item Benutzerfreundlichkeit
        \item Design
        \item Funktionalität
        \item Zuverlässigkeit
        \item Performance
        \item Wartbarkeit- Änderbarkeit
    \end{itemize}
\end{itemize}

\subsection{Aufgabe 1 b;}
    \begin{abstract}
       Laufzeiteigenschaften, Sicherheit Benutzbarkeit, Performance. Hingegen Evolutionseigenschaften nur Testbarkeit, Erweiterbarkeit usw. sind. Siehe Folie 25
        erste Vorlesung.
    \end{abstract}

\subsection{Aufgabe 2 a;}
    \begin{itemize}
        \item 1. Hardware Abstraktion
        \item 2. Hardware-Multiplexing
        \item Hardware-Schutz
    \end{itemize}

\subsection{Aufgabe 2 b;}
    \begin{itemize}
        \item keine Scheduler:
            \begin{itemize}
                \item ein Threadsysteme 
            \end{itemize}
        \item kein Paging:
            \begin{itemize}
                \item 
            \end{itemize}
        \item kein Pozessormodus:
            \begin{itemize}
                \item 
            \end{itemize}
    \end{itemize}

\subsection{Aufgabe 3 a;}   
    \begin{abstract}
        Sparsamkeit bedeutet Eingrenzen von Gegebenheiten (Energie)(Die Eigenschaft eines Systems, seine Funktion mit minimalem
        Ressourcenverbrauch auszuüben.). 
        Effizienz bedeutet egal welchen Verbrauch es benötigt möglichst zuverlässig und schnell zu sein. (Der Grad, zu welchem ein System oder eine seiner Komponenten seine
        Funktion mit minimalem Ressourcenverbrauch ausübt)
    \end{abstract}

\subsection{Aufgabe 3 b;}
    \begin{abstract}
        Sparsamkeit würde ich zu Evolutionseigenschaften zuordnen, da es sich messen lässt. Könnte aber auch unter Begriff Performance fallen.
    \end{abstract}

\subsection{Aufgabe 3 c;}
     \begin{itemize}
        \item Energie:
            \begin{abstract}
                Wenn das Gerät nur eine gewisse Kapatzität mitbringt, aber bei wichtigen Aufgaben sie möglichst schnell erledigt und wenig Strom verbraucht.
            \end{abstract}
        \item Speicherplatz:
            \begin{abstract}
                Bei begrenzten Speicher möglichst alle Blöcke nutzen. Siehe DBS Tabellen.
            \end{abstract} 
     \end{itemize}

\subsection{Aufgabe 3 d;}
     \begin{abstract}
        Sparsamkeit mit Funktionalität bedeutet weniger Funktionen. Kann mit weniger Code erzeugt werden. Linux-Kernel anders als Windows-Kernel.
     \end{abstract}

\subsection{Aufgabe 4 a;}
     \begin{abstract}
        Möglichst guten Akku. Dennoch muss auch gutes und effizientes CPU-Schedulling vorhanden sein. Auch Prefetching kann helfen.
     \end{abstract}

\subsection{Aufgabe 4 b;}
    \begin{itemize}
        \item Reaktivität:
            \begin{abstract}
                Reaktivität abhängig von Energieversorgung (Spannungspegel) aber auch von Software-Gegebenheiten (Prozess-Schedulling, E/A-Schedulling).
            \end{abstract}

        \item Nutzererfahrung:
            \begin{abstract}
              Ist abhängig vom Typ der Anwendung. Kann somit auch von beim abhängen. Siehe Folie 21 Vorlesung 2.  
            \end{abstract}
    \end{itemize}     
